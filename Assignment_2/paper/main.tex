\documentclass[table,xcdraw]{article}

\usepackage{pdfpages}
\usepackage{siunitx}
\usepackage[utf8]{inputenc}
\usepackage{amsmath,mathrsfs}
\usepackage{lmodern}
\usepackage{fullpage}
\usepackage{units}
\usepackage{float}
\usepackage{icomma}
\usepackage{color}
\usepackage{graphicx}
\usepackage{ gensymb }
\usepackage{bbm}
\usepackage{verbatim}
\usepackage[T1]{fontenc}
\usepackage{caption}
\usepackage{tikz}
\usepackage{amssymb}
\usepackage{hyperref}
\usepackage{graphicx}
\usepackage{braket}
\usepackage{wrapfig}
\usepackage[a4paper, total={6in, 9in}]{geometry}
%\usepackage[parfill]{parskip}
%\usepackage[table]{xcolor}
%\usepackage[margin=1in]{geometry}
\usepackage{tabularx}
\usepackage{enumitem}
\usepackage{microtype}
\usepackage{eso-pic}
\usepackage{subcaption}
\usepackage{comment}
\usepackage{units}
\usepackage{pgfplots} 
\usetikzlibrary{backgrounds,automata}
\newcommand{\backgroundpic}[3]{
	\put(#1,#2){
	\parbox[b][\paperheight]{\paperwidth}{
	\centering
	\includegraphics[width=0.8\paperwidth,height=0.8\paperheight,keepaspectratio]{#3}}}}


\usepackage[backend=biber,sortcites,sorting=none]{biblatex}
\addbibresource{references.bib}
\usepackage[swedish, english]{babel}
\usepackage[babel]{csquotes}


\usepackage[toc,page]{appendix}
\renewcommand\appendixtocname{Appendix}
\renewcommand\appendixpagename{Appendix}
\captionsetup[figure]{labelfont=bf,textfont=it,format=hang}

\newcommand{\HRule}{\rule{\linewidth}{0.5mm}}

\setlist{nolistsep}
\definecolor{green}{HTML}{66FF66}
\definecolor{myGreen}{HTML}{009900}
\usepackage{pdfpages}

\usepackage{scrextend}

\usepackage{polynom}

\setlength\parindent{0pt}

\DeclareMathOperator*{\argmax}{arg\,max}



\begin{document}


\begin{titlepage}
\begin{center}

% Upper part of the page. The '~' is needed because \\
% only works if a paragraph has started.

~\\[1.0cm]
 \includegraphics[width=0.3\textwidth]{Figures/chalmers.png}~\\[1.0cm]

\textsc{\LARGE Chalmers tekniska högskola}\\[0.3cm]
TDA231 \textit{Algorithms for Machine Learning \& Interference}\\
Teacher: \textit{Devdatt Dubhashi} \\

% Title
\HRule \\[0.3cm]
{ \huge \bfseries Homework 3 \\[0.3cm] }
\HRule \\[0.3cm]

Victor Nilsson  -  \textit{vicnilss@student.chalmers.se} - 931006-5132\\ [0.3cm]
Bjarki Vilmarsson  -  \textit{bjarkiv@student.chalmers.se} - 920215-0950\\ [0.3cm]


\vfill
% Bottom of the page
{\large \today}

\end{center}
\end{titlepage}

\newpage



\section*{Problem 1}
\subsection*{1.1}

\subsubsection*{a)}

For this classification problem we want to calculate

\begin{equation}
P(c=k|(x_1,\dots,x_n)),
\label{eq:probExpr}
\end{equation}

i.e. if we see the data $\bm{x}=(x_1,\dots,x_n)$ what is the probability that this feature vector belongs to class $k$. Here we only have binary feature vector, i.e. all elements are either $1$ or $0$. We want to estimate the probability $P(c=1|(0,1,1))$ given the previous results:

\begin{equation*}
	\begin{array}{rcl}
		c & = & 1 : (1,1,1),(0,0,1),(1,1,0),(1,0,1) \\
		c & = & 0 : (0,0,0),(1,0,0),(0,0,1),(0,1,0)
	\end{array}
\end{equation*}

For this problem we assume that the features are internally conditionally independent, meaning that 

\begin{equation}
\label{eq:indeProb}
P(x_i,x_j|c=k)=P(x_i|c=k)P(x_j|c=k)\;\forall\;(i,j)\in\{(a,b); a,b \in \mathbb{N},a\neq b, 0 < a \leq n ,0 < b \leq n \}.
\end{equation}

Using Bayes rule we can write the expression in \eqref{eq:probExpr} as

\begin{equation}
	\begin{array}{rcl}
	P(c=k|(x_1,\dots,x_n)) & = & \displaystyle\frac{P(c=k)P((x_1,\dots,x_n)|c=k)}{\sum_{i=1}^{2}P(c=k_i)P((x_1,	\dots,x_n)|c=k_i)} \\[0.5cm]
	& = & \displaystyle\frac{P(c=k)\prod_{j=1}^{n}P(x_j|c=k)}{\sum_{i=0}^{1}P(c=i)\prod_{j=1}^{n}P(x_j|c=i)}
	\end{array}
	\label{eq:bayes}
\end{equation}

and to calculate $P(c=1|(0,1,1))$ we need the conditional probabilites for the feature elements given the different classes, these values can be found in Table~\ref{tab:condProb}.

\begin{table}[H]
\centering
\caption{Conditional probabilities given the class $k$, the probabilities that will be used in later calculations are marked in green and red.}
\label{tab:condProb}
\begin{tabular}{|c|c|c|c|c|c|c|}
\hline
            & \multicolumn{2}{c|}{i=1}                                                         & \multicolumn{2}{c|}{i=2}          & \multicolumn{2}{c|}{i=3}          \\ \hline
x\_i       & 0                                                  & 1                           & 0                           & 1   & 0   & 1                           \\ \hline
P(x\_i|c=1) &  \cellcolor[HTML]{67FD9A}1/4 & 3/4 & 1/2 & \cellcolor[HTML]{67FD9A}1/2 & 1/2 & \cellcolor[HTML]{67FD9A}1/2 \\ \hline
P(x\_i|c=0) & \cellcolor[HTML]{FD6864}3/4                        & 1/4 & 3/4 & \cellcolor[HTML]{FD6864}1/4 & 3/4 & \cellcolor[HTML]{FD6864}1/4 \\ \hline
\end{tabular}
\end{table}

Using also that $P(c=0)=P(c=1)=\frac{1}{2}$ we can finally calculate the probability,

\begin{equation}
	P(c=k|(0,1,1)) = \frac{\frac{1}{2}\frac{1}{4}\frac{1}{2}\frac{1}{2}}{\frac{1}{2}\frac{3}{4}\frac{1}{4}\frac{1}{4}+\frac{1}{2}\frac{1}{4}\frac{1}{2}\frac{1}{2}}= \frac{4}{7}.
\end{equation}


\subsubsection*{b)}

Now we want to calculate the probability of being in class $c=1$ using only a subset of the possible features. To carry out this calculation we simply omit the unknown feature and use Bayes rule again to find the probability for $P(c=k|(0,1,\Colorcancel[red]{x}))$,

\begin{equation}
P(c=k|(0,1))=\frac{\frac{1}{2}\frac{1}{4}\frac{1}{2}}{\frac{1}{2}\frac{3}{4}\frac{1}{4}+\frac{1}{2}\frac{1}{4}\frac{1}{2}}=\frac{4}{5}
\end{equation}


where we again used the values from Table~\ref{tab:condProb}.



\subsection*{1.2}

The features $x_1,x_2 \text{and} x_3$ are not conditionally independent since only 1 of them are active at once, and thus the relation in \eqref{eq:indeProb} can not be used in Equation~\eqref{eq:bayes}. It would be easier to combine these three features to one, namely

\[
    x_1= 
\begin{cases}
    0, & \text{if customer is younger than 20.} \\
    1, & \text{if customer is between 20 and 30 in age.}\\
    2, & \text{if customer is older than 30.}
\end{cases}
\]

So we only get a 2Dimensional feature vector, but not all the elements are a binary variable.
 
\newpage 

\section*{Problem 2}
\subsection*{2.1}

Bayes classifier 

\begin{equation}
P(T_{new}=c|\bm{x}_{new},\bm{X},\bm{t})= \frac{p(\bm{x}_{new}|T_{new}=c,\bm{X},\bm{t})P(T_{new}=c|\bm{X},\bm{t})}{\sum_{c'=1}^{C}p(\bm{x}_{new}|T_{new}=c',\bm{X},\bm{t})P(T_{new}=c'|\bm{X},\bm{t})}
\end{equation}

The same prior distributions so that, 

\begin{equation}
P(T_{new}=c_i|\bm{X},\bm{t}) = P(T_{new}=c_j|\bm{X},\bm{t})\; \forall \; 0 < i,j \leq C
\end{equation}

and we can factor those out and cancel them.

Then we use that we have spherical gaussians, such that,

\begin{equation}
p(\bm{x}_{new}|T_{new}=c,\bm{X},\bm{t}) = \frac{1}{\sqrt{(2\pi\sigma_c^2)^3}}exp{\left(-\frac{1}{2}(\bm{x}_{new}-\bm{\mu}_c)^{T}\sigma_c^2 \mathbb{I} (\bm{x}_{new}-\bm{\mu}_c) \right)}
\end{equation}

\begin{table}[H]
\centering
\caption{The result of makeing 5-fold cross validation of the dataset. }
\label{my-label}
\begin{tabular}{|l|l|l|l|l|l|l|}
\hline
               & Group1 & Group 2 & Group 3 & Group 4 & Group 5 & Averaged Error \\ \hline
SPH Classifier & 0.005  & 0.01    & 0.075   & 0.005   & 0.0025  & 0.012          \\ \hline
New Classifier & 0.005  & 0.0125  & 0.075   & 0.005   & 0.0025  & 0.013          \\ \hline
\end{tabular}
\end{table}

\subsection*{2.2}

\begin{table}[H]
\centering
\caption{The result of making 5-fold crossvalidation for the digits 5 and 8 using the \texttt{$new\_classifier$} classifer for two different feature vectors, one using all the pixel values and the other the row and column variances.}
\label{my-label}
\begin{tabular}{|l|l|l|l|l|l|l|}
\hline
                  & Group1 & Group 2 & Group 3 & Group 4 & Group 5 & Averaged Error \\ \hline
Pixel Features    & 0.0977 & 0.1032  & 0.1027  & 0.1091  & 0.1086  & 0.1043         \\ \hline
Variance Features & 0.0986 & 0.0986  & 0.0955  & 0.100   & 0.0968  & 0.0979         \\ \hline
\end{tabular}
\end{table}

\end{document}